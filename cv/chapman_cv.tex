% Don't like 10pt? Try 11pt or 12pt
\documentclass[10pt]{article}

% This is a helpful package that puts math inside length specifications
\usepackage{calc}


% Simpler bibsection for CV sections
% (thanks to natbib for inspiration)
\makeatletter
\newlength{\bibhang}
\setlength{\bibhang}{1em}
\newlength{\bibsep}
 {\@listi \global\bibsep\itemsep \global\advance\bibsep by\parsep}
\newenvironment{bibsection}%
        {\vspace{-\baselineskip}\begin{list}{}{%
       \setlength{\leftmargin}{\bibhang}%
       \setlength{\itemindent}{-\leftmargin}%
       \setlength{\itemsep}{\bibsep}%
       \setlength{\parsep}{\z@}%
        \setlength{\partopsep}{0pt}%
        \setlength{\topsep}{0pt}}}
        {\end{list}\vspace{-.6\baselineskip}}
\makeatother

% Layout: Puts the section titles on left side of page
\reversemarginpar

%
%         PAPER SIZE, PAGE NUMBER, AND DOCUMENT LAYOUT NOTES:
%
% The next \usepackage line changes the layout for CV style section
% headings as marginal notes. It also sets up the paper size as either
% letter or A4. By default, letter was used. If A4 paper is desired,
% comment out the letterpaper lines and uncomment the a4paper lines.
%
% As you can see, the margin widths and section title widths can be
% easily adjusted.
%
% ALSO: Notice that the includefoot option can be commented OUT in order
% to put the PAGE NUMBER *IN* the bottom margin. This will make the
% effective text area larger.
%
% IF YOU WISH TO REMOVE THE ``of LASTPAGE'' next to each page number,
% see the note about the +LP and -LP lines below. Comment out the +LP
% and uncomment the -LP.
%
% IF YOU WISH TO REMOVE PAGE NUMBERS, be sure that the includefoot line
% is uncommented and ALSO uncomment the \pagestyle{empty} a few lines
% below.
%

%% Use these lines for letter-sized paper
\usepackage[paper=letterpaper,
            %includefoot, % Uncomment to put page number above margin
            marginparwidth=1.2in,     % Length of section titles
            marginparsep=.05in,       % Space between titles and text
            margin=1in,               % 1 inch margins
            includemp]{geometry}

%% Use these lines for A4-sized paper
%\usepackage[paper=a4paper,
%            %includefoot, % Uncomment to put page number above margin
%            marginparwidth=30.5mm,    % Length of section titles
%            marginparsep=1.5mm,       % Space between titles and text
%            margin=25mm,              % 25mm margins
%            includemp]{geometry}

%% More layout: Get rid of indenting throughout entire document
\setlength{\parindent}{0in}

%% This gives us fun enumeration enumerationvironments. compactitem will be nice.
\usepackage{paralist}

%% Reference the last page in the page number
%
% NOTE: comment the +LP line and uncomment the -LP line to have page
%       numbers without the ``of ##'' last page reference)
%
% NOTE: uncomment the \pagestyle{empty} line to get rid of all page
%       numbers (make sure includefoot is commented out above)
%
\usepackage{fancyhdr,lastpage}
\pagestyle{fancy}
%\pagestyle{empty}      % Uncomment this to get rid of page numbers
\fancyhf{}\renewcommand{\headrulewidth}{0pt}
\fancyfootoffset{\marginparsep+\marginparwidth}
\newlength{\footpageshift}
\setlength{\footpageshift}
          {0.5\textwidth+0.5\marginparsep+0.5\marginparwidth-2in}
\lfoot{\hspace{\footpageshift}%
       \parbox{4in}{\, \hfill %
                    \arabic{page} of \protect\pageref*{LastPage} % +LP
%                    \arabic{page}                               % -LP
                    \hfill \,}}

% Finally, give us PDF bookmarks
\usepackage{color,hyperref}
\definecolor{darkblue}{rgb}{0.0,0.0,0.3}
\hypersetup{colorlinks,breaklinks,
            linkcolor=darkblue,urlcolor=darkblue,
            anchorcolor=darkblue,citecolor=darkblue}

%%%%%%%%%%%%%%%%%%%%%%%% End Document Setup %%%%%%%%%%%%%%%%%%%%%%%%%%%%


%%%%%%%%%%%%%%%%%%%%%%%%%%% Helper Commands %%%%%%%%%%%%%%%%%%%%%%%%%%%%

% The title (name) with a horizontal rule under it
%
% Usage: \makeheading{name}
%
% Place at top of document. It should be the first thing.
\newcommand{\makeheading}[1]%
        {\hspace*{-\marginparsep minus \marginparwidth}%
         \begin{minipage}[t]{\textwidth+\marginparwidth+\marginparsep}%
                {\large \bfseries #1}\\[-0.15\baselineskip]%
                 \rule{\columnwidth}{1pt}%
         \end{minipage}}

% The section headings
%
% Usage: \section{section name}
%
% Follow this section IMMEDIATELY with the first line of the section
% text. Do not put whitespace in between. That is, do this:
%
%       \section{My Information}
%       Here is my information.
%
% and NOT this:
%
%       \section{My Information}
%
%       Here is my information.
%
% Otherwise the top of the section header will not line up with the top
% of the section. Of course, using a single comment character (%) on
% empty lines allows for the function of the first example with the
% readability of the second example.
\renewcommand{\section}[2]%
        {\pagebreak[2]\vspace{1.3\baselineskip}%
         \phantomsection\addcontentsline{toc}{section}{#1}%
         \hspace{0in}%
         \marginpar{
         \raggedright \scshape #1}#2}

% An itemize-style list with lots of space between items
\newenvironment{outerlist}[1][\enskip\textbullet]%
        {\begin{itemize}[#1]}{\end{itemize}%
         \vspace{-.6\baselineskip}}

% An environment IDENTICAL to outerlist that has better pre-list spacing
% when used as the first thing in a \section
\newenvironment{lonelist}[1][\enskip--]%
        {\vspace{-\baselineskip}\begin{list}{#1}{%
        \setlength{\partopsep}{0pt}%
        \setlength{\topsep}{0pt}}}
        {\end{list}\vspace{-.6\baselineskip}}

% An itemize-style list with little space between items
\newenvironment{innerlist}[1][\enskip\textbullet]%
        {\begin{compactitem}[#1]}{\end{compactitem}}

% An environment IDENTICAL to innerlist that has better pre-list spacing
% when used as the first thing in a \section
\newenvironment{loneinnerlist}[1][\enskip\textbullet]%
        {\vspace{-\baselineskip}\begin{compactitem}[#1]}
        {\end{compactitem}\vspace{-.6\baselineskip}}

% To add some paragraph space between lines.
% This also tells LaTeX to preferably break a page on one of these gaps
% if there is a needed pagebreak nearby.
\newcommand{\blankline}{\quad\pagebreak[2]}

% Uses hyperref to link DOI
\newcommand\doilink[1]{\href{http://dx.doi.org/#1}{#1}}
\newcommand\doi[1]{doi:\doilink{#1}}

% For \url{SOME_URL}, links SOME_URL to the url SOME_URL
\providecommand*\url[1]{\href{#1}{#1}}
% Same as above, but pretty-prints SOME_URL in teletype fixed-width font
\renewcommand*\url[1]{\href{#1}{\texttt{#1}}}

% For \email{ADDRESS}, links ADDRESS to the url mailto:ADDRESS
\providecommand*\email[1]{\href{mailto:#1}{#1}}
% Same as above, but pretty-prints ADDRESS in teletype fixed-width font
%\renewcommand*\email[1]{\href{mailto:#1}{\texttt{#1}}}

% Dashes for itemize lists
\def\labelitemi{--}

%%%%%%%%%%%%%%%%%%%%%%%% End Helper Commands %%%%%%%%%%%%%%%%%%%%%%%%%%%

%%%%%%%%%%%%%%%%%%%%%%%%% Begin CV Document %%%%%%%%%%%%%%%%%%%%%%%%%%%%

\begin{document}
\makeheading{Brad Chapman}

\section{Contact Information}
%
% NOTE: Mind where the & separators and \\ breaks are in the following
%       table.
%
% ALSO: \rcollength is the width of the right column of the table
%       (adjust it to your liking; default is 1.85in).
%
\newlength{\rcollength}\setlength{\rcollength}{1.85in}%
%
\begin{tabular}[t]{@{}p{\textwidth-\rcollength}p{\rcollength}}
35 Partridge Avenue & 1-617-447-8586 \\
Somerville, MA 02145, USA & \email{chapmanb@50mail.com} \\
\end{tabular}

\section{Education}
%
Ph.D., University of Georgia, Plant Biology, August 2004
\newline
B.S., Michigan State University, Botany and Plant Pathology, May 1999

\section{Community and code}
\begin{lonelist}
  \item Open source community involvement:
    \begin{innerlist}
      \item[] \href{http://bcb.io}{Blog: Code sharing,
          scientific discussion and documentation}
      \item[] \href{http://www.open-bio.org/wiki/Codefest_2017}{Main organizer
        for community collaboration working groups (Codefest) since 2010}
      \item[] \href{http://www.open-bio.org/wiki/BOSC_2017}{Bioinformatics open source conference organizer since 2011}
    \end{innerlist}
  \item Develop \href{https://github.com/chapmanb/bcbio-nextgen}{bcbio},
    providing validated, scalable, community developed variant calling and
    RNA-seq analysis
  \item Daily programming in Python, Clojure, R and Javascript. Familiarity with
    Perl, Java and C++. (code repositories: \href{http://github.com/chapmanb}{GitHub},
    \href{http://bitbucket.org/chapmanb/}{Bitbucket}).
  \item Comfortable analyzing large datasets in local clusters
    and on containerized cloud environments
\end{lonelist}

\section{Experience}
%
\textbf{Harvard T.H. Chan School of Public Health} Research Scientist. Mar 2011-Present
\begin{itemize}
  \item Established external collaborations with partners in both
    academia and industry to enable critical infrastructure development
  \item Embeded bcbio platform within open source and commercial
    analysis platforms using community workflow standards
  \item Provided automated scaling and analysis for large sequencing
    projects, supporting analysis work in the core
  \item Building reference standards and validation to ensure accuracy and
    reproducibility of analyses
  \item Custom research support in collaboration with researchers,
    including variant calling, RNA-seq, detection of low frequency HIV
    populations, shRNA, and transposon insertion analysis.
    (\href{https://github.com/hbc/projects}{code for recent projects}).
\end{itemize}

\textbf{Massachusetts General Hospital} Bioinformatics Specialist II. Sep 2008-Mar 2011
\begin{itemize}
  \item Custom next-generation sequencing analysis in collaboration with
    researchers in the hospital; RNA-seq, ChIP-seq, short RNA, and SNP 
    analyses for a wide variety of organisms including Human, Mouse, Arabidopsis,
    C elegans, and Drosophila.
    (\href{https://github.com/chapmanb/mgh_projects}{code for recent projects}).
  \item Developed LIMS integrating our Illumina sequencing core with 
    the Galaxy platform (\href{https://bitbucket.org/chapmanb/galaxy-central}{code}).
    Included an automated pipeline to process sequencing data, allowing our 
    group to focus on custom analyses while providing researchers with rapid
    access to their raw sequences and alignments
    (\href{https://github.com/chapmanb/bcbb/tree/master/nextgen}{code}).
\end{itemize}

\textbf{Gen9} Bioinformatics consultant. Nov 2009-Oct 2010
\begin{itemize}
  \item[] Part time consultant at synthetic biology startup. Created a server
    automating a novel synthetic construct design approach.
    Provided documentation, training and support for transferring system
    maintenance to internal programming team.
\end{itemize}

\textbf{Codon Devices} Research Scientist, Bioinformatics. 2005-2008
\begin{itemize}
  \item Designed and implemented automated oligo-based gene synthesis
  strategies that drove all revenue generating business. Combined
  existing academic approaches with internal development efforts using
  continuous feedback from production pipeline.
  \item Provided informatics platform for unique synthesis products, including
  large variant libraries and long construct assemblies.
  \item Directed remote programming team in Bangladesh
    to provide an e-commerce website.
\end{itemize}

\textbf{Plate Genome Mapping Laboratory, University of Georgia} PhD student. 1999-2004
\begin{itemize}
  \item Developed phylogentic algorithms for dating duplication events
  in plant species using whole genome comparative data.
  \item Evaluated duplicate gene evolution through analysis of SNP
  accumulation, generating a new hypothesis explaining the prevalence of
  large scale duplications in plants.
\end{itemize}

\textbf{University of Georgia} Teaching assistant. 2002-2003
\begin{itemize}
  \item[] Teaching Assistant -- Introduction to Gene Technology; Bioinformatics Applications.
\end{itemize}

\textbf{USDA Beltsville Agricultural Research Center, Autar Mattoo} Summer research scientist. 1999
\begin{itemize}
  \item[] RNA isolation and detection of transgenic tomatoes by RT-PCR and
  northern hybridization.
\end{itemize}

\textbf{Cornell University, Susan McCouch} Summer research scientist. 1998
\begin{itemize}
  \item[] Sub-cloning and sequencing of rice bacterial artificial chromosomes.
\end{itemize}

\textbf{University of Minnesota, Neil Olszewski} Summer research scientist. 1997
\begin{itemize}
  \item[] Cloning and protein expression in \textit{Escherichia coli}.
\end{itemize}

\textbf{Michigan State University, Mariam Sticklen} Research scientist. 1996-1999
\begin{itemize}
  \item[] Development of transgenic maize by particle bombardment and
  detection via PCR and Southern hybridization.
\end{itemize}

\section{Publications}
\begin{bibsection}
  \item \href{https://doi.org/10.1093/hmg/ddx051}
   {Li A, Hooli B, Mullin K, Tate RE, Bubnys A, Kirchner R, Chapman B, Hofmann
     O, Hide W, Tanzi RE. Silencing of the Drosophila ortholog of SOX5 leads to
     abnormal neuronal development and behavioral impairment. 2017 \textit{Human Molecular Genetics}}

 \item \href{https://peerj.com/articles/3166/}
   {Ahdesmaki M, Chapman BA, Cingolani P, Hofmann O, Sidoruk A, Lai Z, Kazkarov
     G, Rodichenko, Alperovich M, Jenkins D, Carr TH, Stetson D, Doughery B,
     Barrett JC, Johnson JH. Prioritisation of structural variant calls in
     cancer genomes. 2017. \textit{PeerJ}}
  
 \item \href{http://dl.acm.org/citation.cfm?id=3022658}
   {Filippova A, Chapman B, Geiger RS, Herbsleb JD, Kalyanasundaram A, Trainer
     E, Moser A, Stolzfus A. Hacking and Making at Time-Bounded Events: Current
     Trends and Next Steps in Research and Event Design. 2017. \textit{Computer Supported Cooperative Work and Social Computing}}

 \item \href{https://doi.org/10.1093/nar/gkw227}
   {Lai Z, Markovets A, Ahdesmaki M, Chapman B, Hofmann O, McEwen R, Johnson J,
     Dougherty B, Barrett JC, Dry JR. VarDict: a novel and versatile variant
     caller for next-generation sequencing in cancer research. 2016.
     \textit{Nucleic Acids Research}}

 \item \href{https://doi.org/10.1186/s13058-016-0772-7}
    {Lindstrom S, Ablorh A, Chapman B, Gusev A, Chen G, Tuman C, Eliassen H,
      Price AL, Henderson BE, Le Marchand L, Hofmann O, Haiman CA, Kraft P. Deep
      targeted sequencing of 12 breast cancer susceptibility regions in 4611
      women across four different ethnicities. 2016. \textit{Breast Cancer Research}}

  \item \href{https://dx.doi.org/10.12688%2Ff1000research.9663.1}
    {Harris NL, Cock PJA, Chapman B, Fields CJ, Hokamp K, Lapp H, Munoz-Torres
      M, Wiencko H. The 2016 Bioinformatics Open Source Conference (BOSC). 2016.
      \textit{F1000 Research}}

 \item \href{http://bioinformatics.oxfordjournals.org/content/31/2/299.long}
    {Harris NL, Cock PJ, Chapman BA, Goecks J, Hotz HR, Lapp H.
     The Bioinformatics Open Source Conference (BOSC). 2015.
     \textit{Bioinformatics}}

 \item \href{http://www.nature.com/nbt/journal/v32/n3/full/nbt.2835.html}
   {Zook JM, Chapman B, Wang J, Mittelman D, Hofmann O, Hide W, Salit M.
    Integrating human sequence data sets provides a resource of benchmark SNP
    and indel genotype calls. 2014. \textit{Nature Biotechnology}}

  \item \href{http://www.nature.com/nature/journal/v514/n7522/full/nature13824.html}
    {Sun J, Ramos A, Chapman B, Johnnidis JB, Le L, Ho YJ, Klein A, Hofmann O,
     Camargo FD.
     Clonal dynamics of native haematopoiesis. 2014.
     \textit{Nature}}

  \item \href{http://journals.plos.org/plosone/article?id=10.1371/journal.pone.0090485}
    {Li JZ, Chapman B, Charlebois P, Hofmann O, Weiner B, Porter AJ, Samuel R,
     Vardhanabhuti S, Zheng L, Eron J, Taiwo B, Zody MC, Henn MR, Kuritzkes DR,
     Hide W; ACTG A5262 Study Team, Wilson CC, Berzins BI, Acosta EP, Bastow B,
     Kim PS, Read SW, Janik J, Meres DS, Lederman MM, Mong-Kryspin L, Shaw KE,
     Zimmerman LG, Leavitt R, De La Rosa G, Jennings A.
     omparison of illumina and 454 deep sequencing in participants failing
     raltegravir-based antiretroviral therapy. 2014.
     \textit{PLoS One}}

  \item \href{http://www.biomedcentral.com/1471-2105/15/S14/S7}
    {Möller S, Afgan E, Banck M, Bonnal RJ, Booth T, Chilton J, Cock PJ, Gumbel
     M, Harris N, Holland R, Kalaš M, Kaján L, Kibukawa E, Powel DR, Prins P,
     Quinn J, Sallou O, Strozzi F, Seemann T, Sloggett C, Soiland-Reyes S,
     Spooner W, Steinbiss S, Tille A, Travis AJ, Guimera R, Katayama T, Chapman
     BA.
     Community-driven development for computational biology at Sprints,
     Hackathons and Codefests. 2014.
     \textit{BMC Bioinformatics}}

  \item \href{http://journals.plos.org/ploscompbiol/article?id=10.1371/journal.pcbi.1003153}
    {Paila U, Chapman BA, Kirchner R, Quinlan AR.
     GEMINI: integrative exploration of genetic variation and genome
     annotations. 2013. \textit{PLoS Comput Biol}}

  \item \href{http://www.nature.com/nature/journal/v493/n7434/full/nature11779.html}
    {Tabach Y, Billi AC, Hayes GD, Newman MA, Zuk O, Gabel H, Kamath R, Yacoby
     K, Chapman B, Garcia SM, Borowsky M, Kim JK, Ruvkun G.
     Identification of small RNA pathway genes using patterns of phylogenetic
     conservation and divergence. 2013. \textit{Nature}}

  \item \href{http://www.neurology.org/content/80/19/1762.long}
    {Lieber DS, Calvo SE, Shanahan K, Slate NG, Liu S, Hershman SG, Gold NB,
     Chapman BA, Thorburn DR, Berry GT, Schmahmann JD, Borowsky ML, Mueller DM,
     Sims KB, Mootha VK.
     Targeted exome sequencing of suspected mitochondrial disorders. 2013.
     \textit{Neurology}}

  \item \href{http://www.jbiomedsem.com/content/4/1/6}
    {Katayama T et al.
    The 3rd DBCLS BioHackathon: improving life science data integration with
    Semantic Web technologies. 2013. \textit{J Biomed Semantics}}

  \item \href{http://www.biomedcentral.com/1471-2105/13/209}
    {Talevich E, Invergo BM, Cock PJ, Chapman BA.
    Bio.Phylo: A unified toolkit for processing, analyzing and
    visualizing phylogenetic trees in Biopython. 2012.
    \textit{BMC Bioinformatics.}}

  \item \href{http://www.biomedcentral.com/1471-2105/13/315}
    {Afgan E, Chapman B, Taylor J.
     CloudMan as a platform for tool, data, and analysis distribution. 2012.
     \textit{BMC Bioinformatics}}

  \item \href{http://cda.currentprotocols.com/WileyCDA/CPUnit/refId-bi1109.html}
    {Afgan E, Chapman B, Jadan M, Franke V, Taylor J. Using cloud
      computing infrastructure with CloudBioLinux, CloudMan, and
      Galaxy. 2012. \textit{Curr Protoc Bioinformatics.}}

  \item \href{http://www.biomedcentral.com/1471-2105/13/42/}
   {Krampis K, Booth T, Chapman B, Tiwari B, Bicak M, Field D, and Nelson K.
    Cloud BioLinux: pre-configured and on-demand bioinformatics
    computing for the genomics community. 2012. \textit{BMC Bioinformatics}.}

  \item \href{http://www.nature.com/ng/journal/v44/n2/full/ng.1054.html}
    {Sansone SA et al. Toward interoperable bioscience data. 2012.
    \textit{Nature Genetics}.}

  \item \href{http://nar.oxfordjournals.org/content/40/D1/D984.long}
    {Ho Sui SJ, Begley K, Reilly D, Chapman B, McGovern R, Rocca-Sera
      P, Maguire E, Altschuler GM, Hansen TA, Sompallae R, Krivtsov A,
      Shivdasani RA, Armstrong SA, Culhane AC, Correll M, Sansone SA,
      Hofmann O, and Hide W. The Stem Cell Discovery Engine: an
    integrated repository and analysis system for cancer stem cell
    comparisons. 2012. \textit{Nucleic Acids Res}.}

  \item \href{http://www.cell.com/cell-reports/abstract/S2211-1247(11)00018-0}
    {Yuan CC, Matthews AGW, Jin Y, Chen CF, Chapman BA, Ohsumi TK,
      Glass KC, Kutateladze TG, Borowsky ML,
      Struhl K and, Oettinger MA. Histone H3R2 Symmetric Dimethylation
      and Histone H3K4 Trimethylation Are Tightly Correlated in
      Eukaryotic Genomes. 2012. \textit{Cell Reports}.}

  \item \href{http://www.pnas.org/content/108/51/20497.abstract}
    {Simon MD, Wang CI, Kharchenko PV, West JA, Chapman BA,
     Alekseyenko AA, Borowsky ML, Kuroda MI, and Kingston RE.
     The genomic binding sites of a noncoding RNA. 2011.
     \textit{Proc Natl Acad Sci U S A.}.}

  \item \href{http://genesdev.cshlp.org/content/25/20/2210.abstract}
    {Grau DJ, Chapman BA, Garlick JD, Borowsky M, Francis NJ, and Kingston
    RE. Compaction of chromatin by diverse Polycomb group proteins
    requires localized regions of high charge. 2011. \textit{Genes Dev.}}

  \item \href{http://www.biomedcentral.com/1471-2105/11/S12/S4}
    {Afgan E, Baker D, Coraor N, Chapman B, Nekrutenko A, and J Taylor.
    Galaxy CloudMan: delivering cloud compute clusters. 2010. \textit{BMC
    Bioinformatics}.}

  \item \href{http://www.cell.com/chemistry-biology/retrieve/pii/S1074552110004059}
    {Lippow SM, Moon TS, Basu S, Yoon S, Li X, Chapman BA, Robison K, Lipovšek D 
    and KLJ Prather.
    Engineering enzyme specificity using computational
     design of a defined-sequence library. 2010. \textit{Chem Biol}}

   \item \href{http://nar.oxfordjournals.org/content/38/8/2594.long}
     {Blake WJ, Chapman BA, Zindal A, Lee ME, Lippow SM and BM Baynes.
     Pairwise selection assembly for sequence-independent
     construction of long-length DNA. 2010. \textit{Nucleic Acids Res}}

  \item \href{http://bioinformatics.oxfordjournals.org/content/25/11/1422.long}
    {Cock PJA, Antao T, Chang JT, Chapman BA, Cox CJ, Dalke A, Friedberg I,
    Hamelryck T, Kauff F, Wilczynski B and MJL de Hoon.
    Biopython: freely available Python tools for
     computational molecular biology and bioinformatics. 2009.
     \textit{Bioinformatics}}

  \item \href{http://www.google.com/patents/about?id=j2yYAAAAEBAJ}
    {Afeyan N, Church G, Jacobson J, Baynes BM, Nesmith KG, Chapman BA 
    and B Strack-Louge.
    Methods for Assembly of High Fidelity Synthetic
     Polynucleotides. 2006. WO/2006/044956}

   \item \href{http://www.pnas.org/content/103/8/2730.long}
     {Chapman BA, Bowers JE, Feltus FA and AH Paterson.
     Buffering of crucial functions by paleologous
     duplicated genes may contribute cyclicality to angiosperm genome
     duplication. 2006. \textit{Proc Natl Acad Sci USA}}

  \item \href{http://linkinghub.elsevier.com/retrieve/pii/S0168-9525(06)00296-4}
    {Paterson AH, Chapman BA, Kissinger JC, Bower JE, Feltus FA and JC Estill.
    Many gene and domain families have convergent fates
    following independent whole-genome duplication events in
    \textit{Arabidopsis}, \textit{Oryza}, \textit{Saccharomyces} and
    \textit{Tetraodon}. 2006. \textit{Trends Genet}}

  \item \href{http://bioinformatics.oxfordjournals.org/content/20/2/180.abstract}
    {Chapman BA, Bowers JE, Schulze SR and AH Paterson.
    A comparative phylogenetic approach for dating whole genome
    duplication events. 2004. \textit{Bioinformatics}}

  \item \href{http://www.pnas.org/content/101/26/9903.long}
    {Paterson AH, Bowers JE and BA Chapman.
    Ancient polyplodization predating divergence of the
    cereals, and its consequences for comparative genomics. 2004.
    \textit{Proc Natl Acad Sci USA}}

  \item \href{http://linkinghub.elsevier.com/retrieve/pii/S0958166904000308}
    {Paterson AH, Bowers JE, Chapman BA, Peterson DG, Rong J and TM Wicker. 
    Comparative genome analysis of monocots and dicots, towards characterization 
    of angiosperm diversity. 2004. \textit{Curr Opin Biotech}}

  \item \href{http://www.nature.com/nature/journal/v422/n6930/full/nature01521.html}
    {Bowers JE, Chapman BA, Rong J and AH Paterson. 
    Unravelling angiosperm genome evolution by phylogenetic analysis of
    chromosomal duplication events. 2003. \textit{Nature}}

  \item \href{http://linkinghub.elsevier.com/retrieve/pii/S0959437X03001400}
    {Paterson AH, Bowers JE, Peterson DG, Estill JC and BA Chapman.
    Structure and evolution of cereal genomes. 2003. \textit{Curr Opin Genet Devel}}

  \item \href{http://linkinghub.elsevier.com/retrieve/pii/S016894520300061X}
    {Zhong H, Teymouri F, Chapman B, Maqbool S, Sabzikar R, El-Maghraby Y, 
    Dale B and MB Sticklen.
    The dicot pea (\textit{Possum sativum} L.) rbcS transit peptide directs the
    \textit{Alcaligenes eutrophus} polyhydroxybutyrate enzymes into
    the monocot maize (\textit{Zea mays} L.) chloroplasts. 2003.
    \textit{Plant Sci}}

  \item \href{http://dx.doi.org/10.1007/s00122-003-1231-2}
    {Blair MW, Garris AJ, Iyer AS, Chapman BA, Kresovich S and SR McCouch.
    High resolution genetic mapping and candidate gene identification 
    at the \textit{xa5} locus for bacterial blight 
    resistance in rice (\textit{Oryza sativa} L.). 2003. \textit{Theor Appl Genet}}
  
  \item \href{http://www.ucpress.edu/book.php?isbn=9780520231573}
    {Chapman BA, Jacobo CM, Bunch RA and AH Paterson. 
    Cactus breeding and biotechnology. 2002. 
    in \textit{Cacti: Biology and Uses}}
\end{bibsection}

% \section{References}
% %
% Mark Borowsky (\email{borowsky@molbio.mgh.harvard.edu})
% %
% \begin{outerlist}
%   \item[] Mark worked with me for 2 years as director of the Bioinformatics
%     core at Mass General Hospital. He can discuss my current work collaborating
%     with research scientists, analyzing Illumina sequencing projects, and leading
%     Galaxy integration efforts.
% \end{outerlist}

% \blankline

% Stan Letovsky (\email{letovsky@bu.edu})
% %
% \begin{outerlist}
%   \item[] Stan was director of informatics for 2 years while I worked at Codon. He
%     is familiar with my high level interactions and relationships within the company,
%     and ability to coordinate complex projects.
% \end{outerlist}

% \blankline

% Mark Pytlik (\email{mpytlik@gmail.com})
% %
% \begin{outerlist}
%   \item[] Mark and I worked together for 4 years at Codon Devices. He can help with
%     questions about my coding, development work and interactions within a software team.
% \end{outerlist}


\end{document}

%%%%%%%%%%%%%%%%%%%%%%%%%% End CV Document %%%%%%%%%%%%%%%%%%%%%%%%%%%%%
