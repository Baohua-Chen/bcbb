\documentclass{sigchi-ext}
% Please be sure that you have the dependencies (i.e., additional
% LaTeX packages) to compile this example.
\usepackage[T1]{fontenc}
\usepackage{textcomp}
\usepackage[scaled=.92]{helvet} % for proper fonts
\usepackage{graphicx} % for EPS use the graphics package instead
\usepackage{balance}  % for useful for balancing the last columns
\usepackage{booktabs} % for pretty table rules
%\usepackage{ccicons}  % for Creative Commons citation icons
\usepackage{ragged2e} % for tighter hyphenation

% Some optional stuff you might like/need.
% \usepackage{marginnote} 
% \usepackage[shortlabels]{enumitem}
% \usepackage{paralist}
% \usepackage[utf8]{inputenc} % for a UTF8 editor only

%% EXAMPLE BEGIN -- HOW TO OVERRIDE THE DEFAULT COPYRIGHT STRIP --
% \copyrightinfo{Permission to make digital or hard copies of all or
% part of this work for personal or classroom use is granted without
% fee provided that copies are not made or distributed for profit or
% commercial advantage and that copies bear this notice and the full
% citation on the first page. Copyrights for components of this work
% owned by others than ACM must be honored. Abstracting with credit is
% permitted. To copy otherwise, or republish, to post on servers or to
% redistribute to lists, requires prior specific permission and/or a
% fee. Request permissions from permissions@acm.org.\\
% {\emph{CHI'14}}, April 26--May 1, 2014, Toronto, Canada. \\
% Copyright \copyright~2014 ACM ISBN/14/04...\$15.00. \\
% DOI string from ACM form confirmation}
%% EXAMPLE END

% Paper metadata (use plain text, for PDF inclusion and later
% re-using, if desired).  Use \emtpyauthor when submitting for review
% so you remain anonymous.
\def\plaintitle{Collaborative community coding events in open source biological research}
\def\plainauthor{Brad Chapman}
\def\emptyauthor{}
\def\plainkeywords{bioinformatics, collaboration, training, diversity,
  open source, biology}

\title{Collaborative community coding events in open source biological research}

\numberofauthors{1}
% Notice how author names are alternately typesetted to appear ordered
% in 2-column format; i.e., the first 4 autors on the first column and
% the other 4 auhors on the second column. Actually, it's up to you to
% strictly adhere to this author notation.
\author{%
  \alignauthor{%
    \textbf{Brad Chapman}\\
    \affaddr{Harvard Chan School Bioinformatics Core} \\
    \affaddr{\url{http://bioinformatics.sph.harvard.edu/}} \\
    \email{bchapman@hsph.harvard.edu} } }

% Make sure hyperref comes last of your loaded packages, to give it a
% fighting chance of not being over-written, since its job is to
% redefine many LaTeX commands.
\definecolor{linkColor}{RGB}{6,125,233}
\hypersetup{%
  pdftitle={\plaintitle},
%  pdfauthor={\plainauthor},
  pdfauthor={\emptyauthor},
  pdfkeywords={\plainkeywords},
  bookmarksnumbered,
  pdfstartview={FitH},
  colorlinks,
  citecolor=black,
  filecolor=black,
  linkcolor=black,
  urlcolor=linkColor,
  breaklinks=true,
}

% \reversemarginpar%

\begin{document}

%% For the camera ready, use the commands provided by the ACM in the Permission Release Form.
%\CopyrightYear{2007}
%\setcopyright{rightsretained}
%\conferenceinfo{WOODSTOCK}{'97 El Paso, Texas USA}
%\isbn{0-12345-67-8/90/01}
%\doi{http://dx.doi.org/10.1145/2858036.2858119}

\maketitle

% Uncomment to disable hyphenation (not recommended)
% https://twitter.com/anjirokhan/status/546046683331973120
\RaggedRight{}

% Do not change the page size or page settings.
\begin{abstract}
The Open Bioinformatics Foundation Codefest is a multiple day collaborative
working session. Codefest provides a venue for real time collaboration between
researchers who have established relationships through decentralized open source
work, as well as a place for new developers to integrate with a welcoming
community. We'll describe the unique collaborative structure of Codefest and
discuss approaches to help improve the event to sustain long term collaborations
while training a diverse set of attendees.
\end{abstract}

\keywords{\plainkeywords}

\section{Background and motivation}

The Open Bioinformatics Foundation (\url{https://www.open-bio.org}) is a
community of scientists creating open source code to solve biological problems.
A yearly conference, started in 2000, provides the opportunity for in person
discussion and presentation on technical work about code development and
biological analyses.

In 2010, we recognized a need for a more practical hands on working session in
addition to the conference and developed a two day coding event called the
OpenBio Codefest (\url{https://www.open-bio.org/wiki/Codefest}). This event
continued the past 8 years in a wide diversity of locations, with the most
recent taking place at a non-profit, community-run hackerspace in Prague
(\url{https://www.open-bio.org/wiki/Codefest_2017}).

This summer, we've combined with another open source community to create a full
bioinformatics community conference including dedicated training, traditional
conference talks and four days of collaboration
(\url{https://gccbosc2018.sched.com/}).

Our goals at the CHI 2018 Hackathon Workshop are to describe the unique
collaborative structure of Codefest, connect with other organizers building
long term community relationships through collaborative events, and
learn about how we can improve at training a diverse set of attendees.

\section{Collaborative event design}

Codefest initially started as a space for community members who were already
collaborating remotely to sit together and work. Over time, it expanded to better
incorporate new members into the community by serving as a fun and open
environment for sharing work and meeting like-minded researchers.

We plan to share some unique design elements we've learned in organizing Codefest:

\begin{itemize}
\item The value of collaboration over competition. Codefest has no prizes or
  competitive structure, and instead focuses on producing useful practical code that
  we can share at the associated conference and more widely through blog posts
  and scientific papers.

\item The power of self-organizing groups. We do not pre-define the agenda for
  Codefest and let the attendees suggest areas of focus and then provide
  introductions so working groups can form. This allows newer community members
  to work alongside more experienced developers in areas they'd like to learn,
  and to allow the community to shift focus with new technologies and approaches.

\item The advantage of in person discussion for developing interoperability
  standards. One successful outcome of Codefest have been the development of tool
  communication standards which allow different communities to share development
  resources. Like other projects at Codefest, standards creation happened
  organically due to the need for larger projects to be able to better to re-use
  analyses. These standards have been essential for forming new long term
  collaborations for building necessary research infrastructure.
\end{itemize}

\section{Training and community building}

The biological problems we work on at Codefest require collaboration across
a diverse set of research areas. We're
continually focused on strengthening and improving our community and are hoping
to learn from other organizers at the Workshop:

\begin{itemize}
\item How to attract a more diverse set of community members. Like many programming
  and bioinformatics conferences, we struggle to attract a diverse crowd of
  attendees. As a result, Codefest can feel intimidating or unwelcoming to those
  outside the community. We've received universal praise that we're welcoming
  once overcoming that initial hurdle, but would like ways to project this
  welcoming attitude so under-represented researchers feel comfortable investing
  their time and expertise at Codefest.

\item Incorporating teaching and training into the content of Codefest. As we've
  increasingly tried to attract new community members, we've developed the need
  to help integrate them into the community. In many cases, new members will be
  experts in some areas but not in the projects or languages under active
  development at Codefest. We need to develop methods to quickly get them
  comfortable and productive so they can contribute within a reasonably short
  time frame. At this year's upcoming Codefest we plan to evaluate the impact of
  having dedicated training prior to the collaborative hands on event.

\item Scaling events to incorporate new members and approaches. As we actively
  recruit new attendees we're running into the issue of figuring out how to
  support them at larger scale. Our approach of having a few mentors who make
  connections and provide orientation on projects will need improvement if we're
  successful in recruiting new, diverse attendees.
\end{itemize}

Attending the Hackathon Workshop is a chance to share areas where we've been
successful and to learn how to be better organizers. We hope to continue to
expand and improve Codefest and related events for the open bioinformatics
community.

\end{document}

%%% Local Variables:
%%% mode: latex
%%% TeX-master: t
%%% End:
